Веб-приложение состоит из двух основных частей: back-end и front-end. Back-end -- это сам веб-сервер,
который осуществляет обработку запросов пользователей, получение и обработку данных.
Front-end -- это пользовательский интерфейс, визуализирующий полученные от back-end данные в понятный вид.
С помощью этого интерфейса пользователь способен не только получать, но и передавать данные на back-end.

Back-end составляющая сервиса реализована на языке Python 3, с использованием библиотек: Flask~---
обрабатывает HTTP запросы от пользователей, pandas~--- обработка и хранение данных; sklearn~--- модели TFIDF, KMeans, SVM;
pymystem3~--- лемматизация.

Структура приложения показана на рис. \ref{server_struct}.

\begin{figure}[h]
    \centering
    \includegraphics[width=1\textwidth]{server_infrastructure.pdf}
    \caption{Структура сервера}
    \label{server_struct}
\end{figure}

Back-end разделён на несколько модулей: Flask web-сервер, модуль парсеров новостей, которые
с помощью параллельных процессов извлекают данные с сайтов СМИ, модуль анализа данных.

При запуске сервера происходит получение новостей за последние 12 часов и их обработка.
После этого запускается отдельный процесс, отвечающий за актуальность данных: каждую минуту он
проверяет наличие новых статей, и, если такие есть, отправляет их на обработку. При запросе от Front-end, сервер
обращается к уже обработанным данным, хранящимся в кэше.

Основной частью сервера является класс Analizer, обрабатывающий полученные
новостные статьи.

Методы класса:
\begin{itemize}
    \item \textit{Конструктор}. Параметры: список новостных статей с мета-данными.

    В конструкторе класса производится загрузка сохраненных моделей, определяются параметры моделей агрегации, вызываются функции 
    функции \textit{\_data\_to\_pandas}, \textit{\_classify}, \textit{\_classify}, \textit{\_aggregate}, \textit{\_form\_output}.

    \item \textit{append\_data}. Параметры: список новостных статей с мета-данными.

    Функция отвечает за добавление новых данных. Вызываются те же методы, что и
    в конструкторе. Дополнительно удаляются устаревшие данные.

    \item \textit{\_data\_to\_pandas}. Параметры: список новостных статей с мета-данными. Возвращаемое значение: \verb|Pandas Dataframe|, в котором хранится
    вся информация о актуальных новостях.

    Список новостей конвертируется в таблицу \verb|Pandas Dataframe|, из которой
    удаляются дубликаты, лишние мета-данные, происходит сортировка по дате 
    публикации статьи. Вызывается функция \textit{\_norimalize}.

    \item \textit{\_norimalize}. Параметры: \verb|Pandas Dataframe|.

    Вызываются конвеер с функциями нормализации. К таблице
    добавляются нормализованные текст и заголовок статей.

    \item \textit{\_classify}. Параметры: \verb|Pandas Dataframe|.

    Вызываются функции классификации. Для каждого классификатора реализована отдельная функция, в которой к таблице добавляются колонка с тегами,
    полученными от данного классификатора.

    \item \textit{\_aggregate}. Параметры: \verb|Pandas Dataframe| и 
    конфигурация агрегаторов.

    Вызываются функции агрегации. Для каждого агрегатора реализована отдельная функция, 
\end{itemize}
% \subsection{Back-end часть}
% \subsection{Front-end часть}